Asynchronous execution protocols represent a deliberate design tradeoff for scaling blockchain systems, sacrificing global ordering in order to significantly reduce latency and increase throughput. By avoiding consensus in the fast path and restricting state updates to commutative or single-writer operations, these protocols enable confirmations within one or two network round-trips while scaling horizontally with the number of participants. This paradigm underlies a growing class of systems, including fast payment protocols~\cite{fastpay, abc}, single-writer state machines~\cite{sui-lutris, consensus-number}, and CRDT-like BFT objects~\cite{stingray}.

Despite their performance advantages, existing asynchronous protocols suffer from a fundamental liveness limitation in the presence of client faults. In particular, if a client equivocates by issuing multiple conflicting transactions with the same nonce, the affected state can become permanently locked. Such equivocation may not be malicious and can arise naturally from simple crash faults; for example, when a client restarts without remembering that it has already sent a transaction using a given nonce. Prior work addresses this issue by introducing a slower fallback consensus mechanism to recover from these faults~\cite{sui-lutris, consensus-on-demand}. While correct, this approach significantly degrades latency, increases protocol complexity, and undermines the core motivation for asynchronous execution.

In this thesis, we present a fast-path recovery protocol that enables faulty clients to recover from nonce equivocation without relying on fallback consensus. We give a concrete construction that extends a FastPay-style asynchronous protocol~\cite{fastpay} with an explicit recovery mechanism inspired by the Simplex protocol~\cite{simplex, minimmit}. The resulting protocol preserves one round-trip latency, operates under a $5f+1$ fault model, and maintains safety and liveness guarantees. While we instantiate the construction in the context of FastPay, the recovery mechanism applies more broadly to asynchronous protocols that rely on commutative updates, single-writer state, or CRDT-like abstractions~\cite{crdt}. We provide intuition for extending the recovery approach to other fast asynchronous constructions. We implement the proposed protocol and evaluate its performance, showing that fast-path recovery can be achieved with minimal overhead and without reintroducing consensus.
