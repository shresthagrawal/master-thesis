\chapter{Conclusions, Limitations, and Future Work}
\label{chap:conclusion}
\za{this section needs work: you need to add all the limitations, practical considerations, assumptions that could be improved, and speculations on how, which you leave as future work.}
This thesis addressed a fundamental liveness limitation in asynchronous blockchain protocols: the
inability to recover from client equivocation without resorting to fallback consensus. We presented
a fast-path recovery protocol that enables locked accounts to resume transacting while preserving
the low-latency guarantees that motivate asynchronous execution.

Our key insight is that recovery from client equivocation closely mirrors view change in Byzantine
consensus protocols. By operating in the $5f+1$ fault model, we ensure that any two $(n-f)$-certificates
share at least $2f+1$ honest validators, guaranteeing majority support for the same transaction.
This property enables safe recovery: a client can construct a recovery transaction that references
the transaction with majority support, and validators can safely advance the account's nonce without
violating safety.

We gave a concrete construction extending FastPay with an explicit recovery mechanism inspired by
Simplex-style view change. The protocol handles the full range of equivocation scenarios, including
cases where no transaction achieves majority support (resolved by signing $\bot$) and recursive
recovery when recovery transactions themselves conflict. The mechanism is implemented as a special
recovery contract address, maintaining compatibility with existing Ethereum wallets and tooling.

We implemented a prototype of the protocol, demonstrating the feasibility of the construction.
The recovery mechanism integrates naturally with the existing FastPay transaction flow, requiring
only the addition of a recovery contract address and extended certificate handling logic.

